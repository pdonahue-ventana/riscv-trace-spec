\chapter{Future directions} \label{Future}

The current focus is the compressed branch trace, however there a
number of other types of processor trace that would be useful 
(detailed below in no particular order). These
should be considered as possible features that maybe added in the future,
once the current scope has been completed.

\section{Data trace}

The trace encoder will output packets to communicate information
about loads and stores to an off-chip decoder.  To reduce the amount
of bandwidth required, reporting data values will be optional, and
both address and data will be able to be encoded differentially when
it is beneficial to do so.  This entails outputting the difference
between the new value and the previous value of the same transfer
size, irrespective of transfer direction.

Unencoded values will be used for synchronisation and at other times.

\section{Fast profiling}

In this mode the encoder will provide a non-intrusive alternative to
the traditional method of profiling, which requires the processor to
be halted periodically so that the program counter can be sampled.
The encoder will issue packets when an exception, call or return is
detected, to report the next instruction executed (i.e. the
destination instruction).  Optionally, the encoder will also be able to
report the current instruction (i.e. the source instruction).

\section{Inter-instruction cycle counts}

In this mode the encoder will trace where the CPU is stalling by
reporting the number of cycles between successive instruction
retirements.

\section{Using a jump target cache to further improve efficiency}

The encoder could include a small cache of uninferable jump targets, managed using a
least-recently-used (LRU) algorithm.  When an uninferable PC discontinuity occurs, if 
the target address is present in the cache, report the index number of the cache
entry (typically just a few bits) rather than the target address itself.  The decoder 
would need to model the cache in order to know the target address associated with
each cache entry.

\textbf{DISCUSSION POINT}:

This could be reported by using format 0 packets with a payload as follows:

\begin{itemize}
  \item Jump target index number
  \item \textbf{branches} (0 - 31, 0 means no branch map)
  \item \textbf{branch\_map} (if branch count is non-zero)
  \item \textbf{irfail} 
  \item \textbf{irdepth}
\end{itemize}

The last 2 fields allows return addresses that fail the implicit return prediction but 
which reside in the jump target cache to be reported using this format.  An implementation
could omit these if all implicit return failures are reported using format 1.

\section{Branch-map partitioning}

\textbf{DISCUSSION POINT}:

The choice of lengths for the \textbf{branch\_map} field in format 1 packets is currently defined as
1, 9, 17, 25, 31.  We should consider an alternative 'tapered' approach, where the choice is 1, 3, 7, 15, 31.
This should result in improved efficiency, and adds zero hardware cost.

The reasoning here is that on average there will be some 'wasted' bits in a format 1 \textit{te\_inst}
because the number of branches to report is less than the selected size of the \textbf{branch\_map} field.
Using a tapered set of sizes means that the number of wasted bits will on average be less for shorter packets.
If the number of branches between updiscons is randomly distributed then the probabilty of generating packets with large
branch counts will be lower, in which case increased waste for longer packets will have less overall impact.
Furthermore, the rate at which packets are generated can be higher for lower branch counts, and so reducing
waste for this case will improve overall bandwidth at times where it is most important.

\section{Transport}

After the current charter has been satisfied the transport mechanism
should be defined and standardised. This will include Aurora based
serdes, PCIe and Ethernet.
