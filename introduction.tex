\chapter{Introduction}
\label{sec:intro}

In complex systems understanding program behavior is not easy.
Unsurprisingly in such systems, software sometimes does not behave as
expected. This may be due to a number of factors, for example,
interactions with other cores, software, peripherals, realtime
events, poor implementations or some combination of all of the above.

It is not always possible to use a debugger to observe behavior of a
running system as this is intrusive.  Providing visibility of program
execution is important.  This needs to be done without swamping the
system with vast amounts of data.

One method of achieving this is via a Processor Branch Trace.

This works by tracking execution from a known start address and sending
messages about the address deltas taken by the program. These deltas are
typically introduced by jump, call, return and branch type instructions,
although interrupts and exceptions are also types of deltas.

Conceptually, the system has one or more of the following fundamental components:

\begin{itemize}
  \item
    A core with an instruction trace interface that outputs all relevant
    information to successfully create a processor branch trace and more.
    This is a high bandwidth interface: in most implementations, it will supply
    a large amount of data (instruction address, instruction type, context information, ...)
    for each core execution clock cycle;
  \item
    A hardware encoder that connects to this instruction trace interface and compresses
    the information into lower bandwidth trace packets;
  \item
    A transmission channel to transmit or a memory to store these trace packets;
  \item
    A decoder, usually software on an external PC, that takes in the trace
    packets and, with knowledge of the program binary that's running on the
    originating hart, reconstructs the program flow. This decoding step can
    be done off-line or in real-time while the hart is executing.
\end{itemize}

In RISC-V, all instructions are executed unconditionally or at least
their execution can be determined based on the program binary. The
instructions between the deltas can all be assumed to be executed
sequentially. Because of this, there is no need to report sequential 
instructions in the trace, only whether the branches were taken or not
and the address of taken indirect branches or jumps. If the program
counter is changed by an amount that cannot be determined from the
execution binary, the trace decoder needs to be given the destination
address (i.e. the address of the next valid instruction).  Examples of
this are indirect branches or jumps, where the next instruction
address is determined by the contents of a register rather than a
constant embedded in the program binary.

Interrupts generally occur asynchronously to the program's execution
rather than intentionally as a result of a specific instruction or
event.  Exceptions can be thought of in the same way, even though they
can be typically linked back to a specific instruction address.  The
decoder generally does not know where an interrupt occurs in the
instruction sequence, so the trace encoder must report the address
where normal program flow ceased, as well as give an indication of the
asynchronous destination which may be as simple as reporting the
exception type.  When an interrupt or exception occurs, or the
processor is halted, the final instruction retired beforehand must be
included in the trace.

This document serves to specify the ingress port (the signals between
the RISC-V core and the encoder), compressed branch trace algorithm and
the packet format used to encapsulate the compressed branch trace
information.

\section{Terminology} \label{sec:terminology}

The following terms havea specific meaning in this specification.

\begin{itemize}
  \item \textbf{ATB}: Arm trace bus
  \item \textbf{branch}: an instruction which conditionally changes the execution flow
  \item \textbf{CSR}: control/status register
  \item \textbf{decoder}: a piece of software that takes the trace packets emitted by the encoder and 
    reconstructs the execution flow of the code executed by the RISC-V hart
  \item \textbf{delta}: a change in the program counter that is other than the difference between two instructions placed consecutively in memory
  \item \textbf{discontinuity}: another name for 'delta' (see above)
  \item \textbf{ELF}: executable and linkable format
  \item \textbf{encoder}: a piece of hardware that takes in instruction execution information from a RISC-V hart and transforms it into trace packets
  \item \textbf{exception}: an unusual condition occurring at run time associated with an instruction in a RISC-V hart
  \item \textbf{hart}: a RISC-V hardware thread
  \item \textbf{interrupt}: an external asynchronous event that may cause a RISC-V hart to experience an unexpected transfer of control
  \item \textbf{ISA}: instruction set architecture
  \item \textbf{jump}: an instruction which unconditionally changes the execution flow
  \item \textbf{direct jump}: an instruction which unconditionally changes the execution flow by changing the PC by a constant value
  \item \textbf{indirect jump}: an instruction which unconditionally changes the execution flow by changing the PC to a computed value
  \item \textbf{inferable jump}: a jump where the target address is supplied via a constant embedded within the jump opcode
  \item \textbf{uninferable jump}: a jump which is not inferable (see above)
  \item \textbf{LSB}: least significant bit
  \item \textbf{MSB}: most significant bit
  \item \textbf{packet}: the atomic unit of encoded trace information emitted by the encoder
  \item \textbf{PC}: program counter
  \item \textbf{program counter}: a register containing the address of the instruction being executed
  \item \textbf{retire}: the final stage of executing an instruction, when the machine state is updated (sometimes referred to as 'commit' or 'graduate')
  \item \textbf{trap}: the transfer of control to a trap handler caused by either an exception or an interrupt
  \item \textbf{updiscon}: contraction of 'uninferable PC discontinuity'
\end{itemize}

\section{Nomenclature}

In the following sections items in \textbf{bold} are signals or
fields within a packet.

Items in \textbf{\textit{bold italics}} are mnemonics for instructions or CSRs defined in the RISC-V ISA

Items in \textit{italics} with names ending \textit{'\_p'} refer to parameters either built into the
hardware or configurable hardware values.


